\documentclass[twocolumn,a4paper,11pt]{scrartcl}

% Language and font encoding
\usepackage[spanish,es-noshorthands]{babel}
\usepackage[utf8]{inputenc}
\usepackage[T1]{fontenc}

% Other necessary packages
\usepackage{graphicx}
\usepackage{amsmath}
\usepackage{cite}

% Title information
\title{Óptica Geométrica}
\author{Nombre del Autor}
\date{}

\begin{document}

\maketitle

\begin{abstract}
Aquí se incluirá un breve resumen del trabajo realizado, destacando los principales objetivos, métodos y resultados obtenidos en el estudio de óptica geométrica.
\end{abstract}

\section{Objetivos}

Comprobar experimentalmente la validez de la ecuación de lentes delgadas, contrastando resultados de laboratorio con predicciones teóricas. Esto nos permitirá verificar la aplicabilidad de los modelos matemáticos en situaciones reales.

Determinar con precisión la distancia focal de una lente delgada mediante mediciones experimentales y cálculos basados en la ecuación de lentes, profundizando así en las propiedades ópticas de estos sistemas.

Analizar la magnificación de la imagen producida por una lente, comparando mediciones experimentales con valores teóricos para evaluar la precisión de nuestras predicciones.


\section{Marco teórico}
Describe los conceptos, principios y ecuaciones importantes relacionados con la práctica, sin ser demasiado extenso. Además, las ecuaciones aparecen enumeradas y de ser necesario cita alguna información de otro(s) autor(es) que ayuden a comprender la teórica de la práctica.

\section{Diseño experimental}
Especifica todo el equipo utilizado en la práctica de manera detallada; además, agrega una o más imágenes indicando el equipo usado y la forma en la que fue conectado para realizar el experimento.

\section{Resultados y discusión}
Presenta los datos recopilados en el laboratorio y también los resultados que obtuvo de manera apropiada, describiendo la metodología y los procedimientos realizados. Además, discute cada uno de sus resultados obtenidos con una secuencia de ideas apropiada y de ser necesario cita de forma correcta para aportar evidencia en su discusión.


\section{Conclusiones}
Concluye sobre los principales resultados obtenidos, teniendo en cuenta los objetivos de la práctica; además, cada una de las conclusiones manifiesta una secuencia de ideas apropiada.

\bibliographystyle{ieeetr}
\bibliography{referencias}  % Asegúrate de crear un archivo referencias.bib con tus referencias

\end{document}