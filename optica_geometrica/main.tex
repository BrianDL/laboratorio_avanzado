\documentclass[twocolumn,a4paper,11pt]{scrartcl}

% Language and font encoding
\usepackage[spanish,es-noshorthands]{babel}
\usepackage[utf8]{inputenc}
\usepackage[T1]{fontenc}

% Other necessary packages
\usepackage{graphicx}
\usepackage{amsmath}
\usepackage{cite}

% Title information
\title{Óptica Geométrica}
\author{Nombre del Autor}
\date{}

\begin{document}

\maketitle

\begin{abstract}
Aquí se incluirá un breve resumen del trabajo realizado, destacando los principales objetivos, métodos y resultados obtenidos en el estudio de óptica geométrica.
\end{abstract}

\section{Objetivos}
En esta sección se detallarán los objetivos específicos del experimento o estudio de óptica geométrica.

\section{Marco teórico}
Aquí se presentarán los fundamentos teóricos relevantes para el estudio de óptica geométrica, incluyendo conceptos clave y ecuaciones importantes.

\section{Diseño experimental}
Esta sección contendrá una descripción detallada del montaje experimental y el procedimiento seguido durante el estudio de óptica geométrica.

\section{Resultados y discusión}
Aquí se presentarán los resultados obtenidos en el experimento, acompañados de un análisis y discusión de los mismos en el contexto de la óptica geométrica.

\section{Conclusiones}
En esta sección se resumirán las principales conclusiones derivadas del estudio, relacionándolas con los objetivos iniciales y el marco teórico presentado.

\bibliographystyle{ieeetr}
\bibliography{referencias}  % Asegúrate de crear un archivo referencias.bib con tus referencias

\end{document}