\documentclass[twocolumn,a4paper,11pt]{scrartcl}

% Language and font encoding
\usepackage[spanish,es-noshorthands]{babel}
\usepackage[utf8]{inputenc}
\usepackage[T1]{fontenc}

% Other necessary packages
\usepackage{graphicx}
\usepackage{amsmath}
\usepackage{cite}
\usepackage{url}
\usepackage{hyperref}

% Additional formatting for two-column layout with centered abstract
\renewcommand{\absnamepos}{empty}  % Remove space where "Abstract" title was
\addto{\captionsspanish}{\renewcommand{\abstractname}{}} % quitar título del resumen

% Title information
\title{El experimento de Rutherford}
\author{Nombre del Autor}
\date{}

\begin{document}

\twocolumn[
  \begin{@twocolumnfalse}
    \maketitle
    \begin{abstract}
    \begin{center}
    \begin{minipage}{0.6\textwidth}
      Describe de manera concisa el trabajo realizado en la práctica, indicando qué se hizo, cómo se hizo y a qué resultados se llegó.
    \end{minipage}
    \end{center}
    \end{abstract}
  \end{@twocolumnfalse}
]

\section{Objetivos}
1. Observar los espectros discretos de radiación γ de fuentes radiactivas de 137 Cs y de 152 Eu.

2. Utilizar el espectro de 152 Eu para calibrar en energı́a el sistema de captura de datos.

3. Verificar la energı́a del pico de 137 Cs.

4. Obtener un indicador de la resolución en energı́a del sistema de detección a 662 KeV.

\section{Marco teórico}
Describe los conceptos, principios y ecuaciones importantes relacionados con la práctica, sin ser demasiado extenso. Además, las ecuaciones aparecen enumeradas y de ser necesario cita alguna información de otro(s) autor(es) que ayuden a comprender la teórica de la práctica.

\section{Diseño experimental}
\section{Calibración en Energía}

El primer paso en la calibración energética implica visualizar los datos adquiridos. Creamos un gráfico de canal vs. conteos utilizando los datos obtenidos del espectro de $^{152}$Eu. Este gráfico proporciona una representación directa de la distribución de la señal a través de los canales del detector.  Posteriormente, comparamos este gráfico generado con el espectro de referencia representado en la Figura 1. Construimos una tabla que liste sistemáticamente los canales correspondientes a los conteos máximos de cada pico de energía observado en la Figura 1. Notamos cuidadosamente que los conteos en el espectro de referencia (Figura 1) se presentan en una escala logarítmica, lo cual debe considerarse durante esta comparación.  A continuación, realizamos un análisis de correlación lineal sobre los datos tabulados en la tabla construida anteriormente. Finalmente, reproducimos el gráfico del espectro de $^{152}$Eu, pero esta vez mostrando la energía en el eje x y los conteos en el eje y, permitiendo una visualización directa de la forma espectral en el espacio de energía.

\section{Energía del Pico del $^{137}$Cs}

Una vez establecida la calibración energética, procedemos a analizar el espectro de $^{137}$Cs. Generamos un gráfico de energía vs. conteos utilizando los datos del espectro de $^{137}$Cs, aplicando la calibración determinada previamente. Después de crear el gráfico, identificamos el pico correspondiente a la señal de $^{137}$Cs. Determinamos con precisión el valor de energía en el punto máximo del pico. Verificamos que esta energía medida coincida estrechamente con la energía conocida del fotón gamma emitido por $^{137}$Cs como se describe en la Figura 2. Para evaluar el rendimiento del sistema de detección de radiación ionizante, calculamos el Ancho Total a Media Altura (FWHM). El FWHM, una medida estándar de la resolución del detector, representa el ancho del pico a la mitad de su altura máxima y se especifica típicamente a una energía dada. Calculamos el FWHM para el pico de $^{137}$Cs para cuantificar la capacidad del detector para resolver características espectrales estrechamente espaciadas.

\section{Resultados y discusión}
Presenta los datos recopilados en el laboratorio y también los resultados que obtuvo de manera apropiada, describiendo la metodología y los procedimientos realizados. Además, discute cada uno de sus resultados obtenidos con una secuencia de ideas apropiada y de ser necesario cita de forma correcta para aportar evidencia en su discusión.


\section{Conclusiones}
Concluye sobre los principales resultados obtenidos, teniendo en cuenta los objetivos de la práctica; además, cada una de las conclusiones manifiesta una secuencia de ideas apropiada.

\bibliographystyle{ieeetr}
\bibliography{referencias} 

\end{document}